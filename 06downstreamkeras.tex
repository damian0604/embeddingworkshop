% !TeX document-id = {f19fb972-db1f-447e-9d78-531139c30778}
% !BIB program = biber
\documentclass[compress]{beamer}
\usepackage[T1]{fontenc}
\usepackage{pifont}
\usetheme[block=fill,subsectionpage=progressbar,sectionpage=progressbar]{metropolis} 

\usepackage{wasysym}
\usepackage{etoolbox}
\usepackage[utf8]{inputenc}

\usepackage{threeparttable}
\usepackage{subcaption}

\usepackage{tikz-qtree}
\setbeamercovered{still covered={\opaqueness<1->{5}},again covered={\opaqueness<1->{100}}}


\usepackage{listings}

\lstset{
	basicstyle=\scriptsize\ttfamily,
	columns=flexible,
	breaklines=true,
	numbers=left,
	%stepsize=1,
	numberstyle=\tiny,
	backgroundcolor=\color[rgb]{0.85,0.90,1}
}



\lstnewenvironment{lstlistingoutput}{\lstset{basicstyle=\footnotesize\ttfamily,
		columns=flexible,
		breaklines=true,
		numbers=left,
		%stepsize=1,
		numberstyle=\tiny,
		backgroundcolor=\color[rgb]{.7,.7,.7}}}{}


\lstnewenvironment{lstlistingoutputtiny}{\lstset{basicstyle=\tiny\ttfamily,
		columns=flexible,
		breaklines=true,
		numbers=left,
		%stepsize=1,
		numberstyle=\tiny,
		backgroundcolor=\color[rgb]{.7,.7,.7}}}{}



\usepackage[american]{babel}
\usepackage{csquotes}
\usepackage[style=apa, backend = biber]{biblatex}
\DeclareLanguageMapping{american}{american-UoN}
\addbibresource{../bdaca/bdaca.bib }
\renewcommand*{\bibfont}{\tiny}

\usepackage{tikz}
\usetikzlibrary{shapes,arrows,matrix}
\usepackage{multicol}

\usepackage{subcaption}

\usepackage{booktabs}
\usepackage{graphicx}

\graphicspath{{../bdaca/pictures/}}

\makeatletter
\setbeamertemplate{headline}{%
	\begin{beamercolorbox}[colsep=1.5pt]{upper separation line head}
	\end{beamercolorbox}
	\begin{beamercolorbox}{section in head/foot}
		\vskip2pt\insertnavigation{\paperwidth}\vskip2pt
	\end{beamercolorbox}%
	\begin{beamercolorbox}[colsep=1.5pt]{lower separation line head}
	\end{beamercolorbox}
}
\makeatother



\setbeamercolor{section in head/foot}{fg=normal text.bg, bg=structure.fg}



\newcommand{\question}[1]{
	\begin{frame}[plain]
		\begin{columns}
			\column{.3\textwidth}
			\makebox[\columnwidth]{
				\includegraphics[width=\columnwidth,height=\paperheight,keepaspectratio]{mannetje.png}}
			\column{.7\textwidth}
			\large
			\textcolor{orange}{\textbf{\emph{#1}}}
		\end{columns}
\end{frame}}

\newcommand{\instruction}[1]{\emph{\textcolor{gray}{[#1]}}}




\title{Beyond Counting Words: Working with Word Embeddings}
\author[Damian Trilling]{Damian Trilling \\ ~ \\ \footnotesize{d.c.trilling@uva.nl \\@damian0604} \\ \url{www.damiantrilling.net}}
\date{12--13 April 2021}
\institute[UvA]{Afdeling Communicatiewetenschap \\Universiteit van Amsterdam}

\begin{document}

\begin{frame}{}
	\titlepage
\end{frame}

\begin{frame}{This part: Keras}
	\tableofcontents
\end{frame}

\begin{frame}[standout]
Disclaimer: I cannot give a full overview of the whole topic of deep learning here -- that's a whole (extensive) course in itself. But embeddings are closely related, that's why we at least will at least get out feet wet a bit.
\end{frame}


\setbeamercovered{transparent}

\section{Neural networks}


\begin{frame}{Neural Networks}
	\begin{itemize}
		\item In ``classical'' machine learning, we predict an outcome directly based on the input features
		\item In neural networks, we can have ``hidden layers'' that we predict
		\item These layers are not necessarily interpretable
		\item ``Neurons'' that ``fire'' based on an ``activation function''
	\end{itemize}
	
\end{frame}

\begin{frame}
	
	\def\layersep{2.5cm}
	
	\begin{tikzpicture}[shorten >=1pt,->,draw=black!50, node distance=\layersep]
	\tikzstyle{every pin edge}=[<-,shorten <=1pt]
	\tikzstyle{neuron}=[circle,fill=black!25,minimum size=17pt,inner sep=0pt]
	\tikzstyle{input neuron}=[neuron, fill=green!50];
	\tikzstyle{output neuron}=[neuron, fill=red!50];
	\tikzstyle{hidden neuron}=[neuron, fill=blue!50];
	\tikzstyle{annot} = [text width=4em, text centered]
	
	% Draw the input layer nodes
	\foreach \name / \y in {1,...,4}
	% This is the same as writing \foreach \name / \y in {1/1,2/2,3/3,4/4}
	\node[input neuron, pin=left:Input \#\y] (I-\name) at (0,-\y) {};
	
	% Draw the hidden layer nodes
	\foreach \name / \y in {1,...,5}
	\path[yshift=0.5cm]
	node[hidden neuron] (H-\name) at (\layersep,-\y cm) {};
	
	
	% Draw the output layer node
	\node[output neuron,pin={[pin edge={->}]right:Output}, right of=H-3] (O) {};
	
	% Connect every node in the input layer with every node in the
	% hidden layer.
	\foreach \source in {1,...,4}
	\foreach \dest in {1,...,5}
	\path (I-\source) edge (H-\dest);
	
	% Connect every node in the hidden layer with the output layer
	\foreach \source in {1,...,5}
	\path (H-\source) edge (O);
	
	% Annotate the layers
	\node[annot,above of=H-1, node distance=1cm] (hl) {Hidden layer};
	\node[annot,left of=hl] {Input layer};
	\node[annot,right of=hl] {Output layer};
	\end{tikzpicture}
	
	$\Rightarrow$ If we had multiple hidden layers in a row, we'd call it a \emph{deep} network.
\end{frame}




\begin{frame}{Why neural networks?}
	\begin{itemize}
		\item learn hidden structures (e.g., embeddings (!))
		\item go beyond the idea that there is a direct relationship between occurrence of word X and label (or occurrence of pixel [R,G,B] and a label)
		\item images, machine translation --- and more and more general NLP, sentiment analysis, etc.
	\end{itemize}
	
	\small {Example of a comparatively easy introduction:
		\url{https://towardsdatascience.com/neural-network-embeddings-explained-4d028e6f0526}}
	
\end{frame}


\begin{frame}[fragile]{Simple feed forward network}
\begin{lstlisting}
model.add(Dense(300, input_dim=input_dim, activation='relu'))
model.add(Dense(1, activation='sigmoid'))
\end{lstlisting}	
	
\begin{itemize}[<+->]
	\item Our first layer reduces the input features (e.g., the 10,000 features our CountVectorizer creates) to 300 neurons
	\item It does so using the relu function $f(x) = max(0, x)$ (as our counts cannot be negartive, just a linear function)
	\item The second layer reduces the 300 neurons to 1 output neuron using the sigmoid function (the S curve you know fron logistic regression)
	\item Of course, we can add multiple layers in between if we want to
\end{itemize}
\end{frame}





\begin{frame}{Convolutional networks}
	The problem with such a basic networks: just as with classic SML, we still loose all information about order (the ``not good'' problem).
	
	Therefore,
\begin{itemize}
	\item We concatenate the vectors of neighboring words
	\item We apply some filter (essentially, we detect patterns)
	\item and then pool the results (e.g., taking the maximum)
\end{itemize}
This means that we now excplitly take into acount \emph{the temporal structure} of a sentence.
\end{frame}



\begin{frame}{Convolutional networks}

	\makebox[\columnwidth]{
	\includegraphics[width=\columnwidth,height=.8\paperheight,keepaspectratio]{ch09_cnn_cropped}}
\end{frame}



\begin{frame}[fragile]{Convolutional networks}
	\begin{lstlisting}
model.add(Embedding(input_dim=vocab_size, output_dim=embedding_dim, input_length=maxlen))
model.add(Conv1D(embedding_dim, 5, activation='relu'))
model.add(GlobalMaxPooling1D())
model.add(Dense(300, activation='relu'))
model.add(Dense(1, activation='sigmoid'))

	\end{lstlisting}	
The layers:	
	\begin{enumerate}[<+->]
		\item train an embedding model
		\item apply the convolution with 5 ``timestamps''
		\item pool using the maximum
		\item another layer with 300 dimensions
		\item the final layer with 1 output neuron
	\end{enumerate}
\end{frame}


\begin{frame}{Convolutional networks}
\textbf{Note that the preprocessing differs!}

\begin{itemize}
	\item We do not take a word vector per document as input any more, but \emph{a sequence of words}
	\item For concatenating, these sequences need to have equal length, which is why we \emph{pad} then
\end{itemize}

\end{frame}




\begin{frame}{LSTM (long short-term memory)}
\begin{itemize}
	\item Unlike ``feed forward'' neural networks, this is  a ``recurrent neural network'' (RNN) -- the training works in two directions
	\item Heavy in computation, very useful for predicting \emph{sequences}
	\item Won't cover today
\end{itemize}
\end{frame}


\section{Using pretrained embeddings}

\begin{frame}{The embedding layer}
\begin{itemize}
	\item Often, the first layer is creating word embeddings
	\item Good embeddings need a lot of training data
	\item Training good embeddings needs time
	\item Therefore, we can replace that layer with a pre-trained embedding layer (!)
	\item We can even use a hybrid approach and allow the pre-trained embedding layer to be re-trained!
\end{itemize}
\end{frame}







\begin{frame}[plain]
	\printbibliography
\end{frame}


\end{document}
