% !TeX document-id = {f19fb972-db1f-447e-9d78-531139c30778}
% !BIB program = biber
\documentclass[compress]{beamer}
\usepackage[T1]{fontenc}
\usepackage{pifont}
\usetheme[block=fill,subsectionpage=progressbar,sectionpage=progressbar]{metropolis} 

\usepackage{wasysym}
\usepackage{etoolbox}
\usepackage[utf8]{inputenc}

\usepackage{threeparttable}
\usepackage{subcaption}

\usepackage{tikz-qtree}
\setbeamercovered{still covered={\opaqueness<1->{5}},again covered={\opaqueness<1->{100}}}


\usepackage{listings}

\lstset{
	basicstyle=\scriptsize\ttfamily,
	columns=flexible,
	breaklines=true,
	numbers=left,
	%stepsize=1,
	numberstyle=\tiny,
	backgroundcolor=\color[rgb]{0.85,0.90,1}
}



\lstnewenvironment{lstlistingoutput}{\lstset{basicstyle=\footnotesize\ttfamily,
		columns=flexible,
		breaklines=true,
		numbers=left,
		%stepsize=1,
		numberstyle=\tiny,
		backgroundcolor=\color[rgb]{.7,.7,.7}}}{}


\lstnewenvironment{lstlistingoutputtiny}{\lstset{basicstyle=\tiny\ttfamily,
		columns=flexible,
		breaklines=true,
		numbers=left,
		%stepsize=1,
		numberstyle=\tiny,
		backgroundcolor=\color[rgb]{.7,.7,.7}}}{}



\usepackage[american]{babel}
\usepackage{csquotes}
\usepackage[style=apa, backend = biber]{biblatex}
\DeclareLanguageMapping{american}{american-UoN}
\addbibresource{../bdaca/bdaca.bib }
\renewcommand*{\bibfont}{\tiny}

\usepackage{tikz}
\usetikzlibrary{shapes,arrows,matrix}
\usepackage{multicol}

\usepackage{subcaption}

\usepackage{booktabs}
\usepackage{graphicx}

\graphicspath{{../bdaca/pictures/}}

\makeatletter
\setbeamertemplate{headline}{%
	\begin{beamercolorbox}[colsep=1.5pt]{upper separation line head}
	\end{beamercolorbox}
	\begin{beamercolorbox}{section in head/foot}
		\vskip2pt\insertnavigation{\paperwidth}\vskip2pt
	\end{beamercolorbox}%
	\begin{beamercolorbox}[colsep=1.5pt]{lower separation line head}
	\end{beamercolorbox}
}
\makeatother



\setbeamercolor{section in head/foot}{fg=normal text.bg, bg=structure.fg}



\newcommand{\question}[1]{
	\begin{frame}[plain]
		\begin{columns}
			\column{.3\textwidth}
			\makebox[\columnwidth]{
				\includegraphics[width=\columnwidth,height=\paperheight,keepaspectratio]{mannetje.png}}
			\column{.7\textwidth}
			\large
			\textcolor{orange}{\textbf{\emph{#1}}}
		\end{columns}
\end{frame}}

\newcommand{\instruction}[1]{\emph{\textcolor{gray}{[#1]}}}




\title{Beyond Counting Words: Working with Word Embeddings}
\author[Damian Trilling]{Damian Trilling \\ ~ \\ \footnotesize{d.c.trilling@uva.nl \\@damian0604} \\ \url{www.damiantrilling.net}}
\date{12--13 April 2021}
\institute[UvA]{Afdeling Communicatiewetenschap \\Universiteit van Amsterdam}

\begin{document}

\begin{frame}{}
	\titlepage
\end{frame}

\begin{frame}{This part: Machine Learning in Python}
	\tableofcontents
\end{frame}



\setbeamercovered{transparent}





\section{Machine learning for textual data}


\begin{frame}[plain]
\makebox[\linewidth]{
	\includegraphics[width=\paperwidth,height=\paperheight,keepaspectratio]{boumanstrilling2016}}
\cite{Boumans2016}
\end{frame}


\begin{frame}{Some considerations}
\begin{itemize}[<+->]
	\item Both can have a place in your workflow (e.g., bottom-up as first exploratory step)
	\item You have a clear theoretical expectation? Bottom-up makes little sense.
	\item But in any case: you need to transform your text into something ``countable''.
\end{itemize}
\end{frame}


\begin{frame}[standout]
There are two main libraries: \textcolor{orange}{scikit-learn} and \textcolor{orange}{gensim}. 

scikit-learn is \textit{the} module for almost all ``classic'' machine learning tasks.

gensim is a specialized module for topic models and embeddings


\tiny
\instruction{show \url{https://scikit-learn.org}}

\instruction{show \url{https://radimrehurek.com/gensim/} and \url{https://github.com/RaRe-Technologies/gensim}}


\end{frame}



	
\section{From text to feature: count vectorizers and tf-idf vectorizers}



\begin{frame}{What is a vectorizer}
\begin{itemize}[<+->]
	\item Transforms a list of texts into a sparse (!) matrix (of word frequencies)
	\item Vectorizer needs to be ``fitted'' to the training data (learn which words (features) exist in the dataset and assign them to columns in the matrix)
	\item Vectorizer can then be re-used to transform other datasets 
\end{itemize}
\end{frame}


\begin{frame}{Different vectorizers}
\begin{enumerate}[<+->]
	\item CountVectorizer (=simple word counts)
	\item TfidfVectorizer (word counts (``term frequency'') weighted by number of documents in which the word occurs at all (``inverse document frequency''))
\end{enumerate}

\pause
$$tfidf_{t,d} = tf_{t,d} \cdot idf_{t}$$

There are different ways to weigh the idf score. A common one is taking the logarithm:

$$idf_{t} = \log \frac{N}{n_t}$$

where $N$ is the total number of documents and $n_t$ is the number of documents containing term $t$
\end{frame}

\begin{frame}{Different vectorizer options}
\begin{itemize}
\item Preprocessing (e.g., stopword removal)
\item Remove words below a specific threshold (``occurring in less than $n=5$ documents'') $\Rightarrow$ spelling mistakes etc.
\item Remove words above a specific threshold (``occuring in more than 50\% of all documents) $\Rightarrow$ de-facto stopwords
\item Not only to improve prediction, but also performance (can reduce number of features by a huge amount)
\end{itemize}
\end{frame}


\begin{frame}[fragile]{Using a scikit-learn vectorizer}
\begin{lstlisting}
from sklearn.feature_extraction.text import CountVectorizer
texts = ['This is the first text text text first', 'And another text yeah yeah']
vec = CountVectorizer(texts)
vec.fit_transform(texts) 

# if we want to see what it looks like
# DON'T DO THIS WITH LARGE MATRICES!
print(vec.get_feature_names())
print(vec.transform(texts).todense()) 
\end{lstlisting}

\instruction{show in notebook}

\end{frame}



\begin{frame}{Before we can do machine learning, we need to make features}

\begin{itemize}[<+->]
	\item typically, (weighted) word frequencies (count vs tf$\cdot$idf)
	\item normalization steps first (lowercasing, punctuation, (stemming/lemmatizing))
	\item potentially also other feature (e.g., named entities -- or only specific word types)
	\item unigrams vs ngrams
	\item pruning (removing extremes)
\end{itemize}
\end{frame}



\section{Classical Machine Learning}



\begin{frame}{Some terminology }
	\begin{columns}[t]
		\column{.5\textwidth}
		
		\begin{block}<1-4>{Supervised machine learning}
			You have a dataset with both predictor and outcome (independent and dependent variables; features and labels) --- a \emph{labeled} dataset.
			\onslide<2>{
				\footnotesize{Think of regression: You measured \texttt{x1}, \texttt{x2}, \texttt{x3} and you want to predict \texttt{y}, which you also measured}}
		\end{block}
		
		\column{.5\textwidth}
		
		\begin{block}<3->{Unsupervised machine learning}
			You have no labels. \onslide<4>{(\footnotesize{You did not measure \texttt{y})}}\\
			\onslide<5>{\textbf{Again, you already know some techniques to find out how \texttt{x1}, \texttt{x2},\ldots \texttt{x\_i} co-occur from other courses:} \begin{itemize}
					\item Principal Component Analysis (PCA) and Singular Value Decomposition (SVD)
					\item Cluster analysis
					\item Topic modelling (Non-negative matrix factorization and Latent Dirichlet Allocation)
					\item \ldots
				\end{itemize}
			}
		\end{block}
		
	\end{columns}
	
\end{frame}


\begin{frame}{Let's distinguish four use cases\ldots}
	
	\begin{enumerate}
		\item Finding similar variables (dimensionality reduction) -- unsupervised
		\item Finding similar cases (clustering) -- unsupervised
		\item Predicting a continous variable (regression) -- supervised
		\item Predicting group membership (classification) -- supervised
	\end{enumerate}
\end{frame}


\begin{frame}[plain]
	\begin{table}[]
		\resizebox{\textwidth}{!}{%
			\begin{tabular}{lllllll}
				& x1 & x2 & x3 & x4 & x5 & y \\
				case1 & \ding{110}  & \ding{110}  & \ding{110}  & \ding{110}  & \ding{110} & \ding{110} \\
				case2 & \ding{110}  & \ding{110}  & \ding{110}  & \ding{110}  & \ding{110} & \ding{110}\\
				case3 & \ding{110}  & \ding{110}  & \ding{110}  & \ding{110}  & \ding{110} & \ding{110}\\
				case4 & \ding{110}  & \ding{110}  & \ding{110}  & \ding{110}  & \ding{110} & \ding{110}\\
			\end{tabular}%
		}
	\end{table}
\end{frame}



\begin{frame}[plain]
	\begin{table}[]
		\resizebox{\textwidth}{!}{%
			\begin{tabular}{lllllll}
				& \textcolor{orange}{x1} & x2 & \textcolor{orange}{x3}& \textcolor{blue}{x4} & \textcolor{blue}{x5} & \textcolor{gray}{(y)} \\
				case1 & \textcolor{orange}{\ding{110}}  & \ding{110}  & \textcolor{orange}{\ding{110}}  & \textcolor{blue}{\ding{110}} & \textcolor{blue}{\ding{110}} & \textcolor{gray}{(\ding{110})} \\
				case2 & \textcolor{orange}{\ding{110}}  & \ding{110}  & \textcolor{orange}{\ding{110}}  & \textcolor{blue}{\ding{110}} & \textcolor{blue}{\ding{110}} & \textcolor{gray}{(\ding{110})} \\
				case3 & \textcolor{orange}{\ding{110}}  & \ding{110}  & \textcolor{orange}{\ding{110}}  & \textcolor{blue}{\ding{110}} & \textcolor{blue}{\ding{110}} & \textcolor{gray}{(\ding{110})} \\
				case4 & \textcolor{orange}{\ding{110}}  & \ding{110}  & \textcolor{orange}{\ding{110}}  & \textcolor{blue}{\ding{110}} & \textcolor{blue}{\ding{110}} & \textcolor{gray}{(\ding{110})} \\
			\end{tabular}%
		}
	\end{table}
	Dimensionality reduction: finding similar variables (features)
\end{frame}


\begin{frame}[plain]
	\begin{table}[]
		\resizebox{\textwidth}{!}{%
			\begin{tabular}{lllllll}
				& x1 & x2 & x3 & x4 & x5 & \textcolor{gray}{(y)} \\
				\textcolor{orange}{case1} & \textcolor{orange}{\ding{110}}  & \textcolor{orange}{\ding{110}}  &\textcolor{orange}{\ding{110}}  &\textcolor{orange}{\ding{110}}   & \textcolor{orange}{\ding{110}} & \textcolor{gray}{(\ding{110})} \\
				\textcolor{blue}{case2} & \textcolor{blue}{\ding{110}}  & \textcolor{blue}{\ding{110}}  &\textcolor{blue}{\ding{110}}  &\textcolor{blue}{\ding{110}}   & \textcolor{blue}{\ding{110}} & \textcolor{gray}{(\ding{110})} \\
				\textcolor{orange}{case3} & \textcolor{orange}{\ding{110}}  & \textcolor{orange}{\ding{110}}  &\textcolor{orange}{\ding{110}}  &\textcolor{orange}{\ding{110}}   & \textcolor{orange}{\ding{110}} & \textcolor{gray}{(\ding{110})} \\
				\textcolor{orange}{case4} & \textcolor{orange}{\ding{110}}  & \textcolor{orange}{\ding{110}}  &\textcolor{orange}{\ding{110}}  &\textcolor{orange}{\ding{110}}   & \textcolor{orange}{\ding{110}} & \textcolor{gray}{(\ding{110})} \\
			\end{tabular}%
		}
	\end{table}
	Clustering: finding similar cases
\end{frame}



\begin{frame}[plain]
	\begin{table}[]
		\resizebox{\textwidth}{!}{%
			\begin{tabular}{llllllll}
				& x1 & x2 & x3 & x4 & x5 & $\rightarrow$ & \textcolor{orange}{y} \\
				case1 & \ding{110}  & \ding{110}  & \ding{110}  & \ding{110}  & \ding{110} & $\rightarrow$ &\textcolor{orange}{\ding{110}}  \\
				case2 & \ding{110}  & \ding{110}  & \ding{110}  & \ding{110}  & \ding{110} & $\rightarrow$ &\textcolor{orange}{\ding{110}} \\
				case3 & \ding{110}  & \ding{110}  & \ding{110}  & \ding{110}  & \ding{110} & $\rightarrow$ &\textcolor{orange}{\ding{110}} \\
				case4 & \ding{110}  & \ding{110}  & \ding{110}  & \ding{110}  & \ding{110} & $\rightarrow$ &\textcolor{orange}{\ding{110}} \\
				&&&&&&& \\
				new case & \ding{110}  & \ding{110}  & \ding{110}  & \ding{110}  & \ding{110} & $\rightarrow$ &\textbf{\textcolor{orange}{?}} \\
			\end{tabular}%
		}
		Regression and classification: learn how to predict $y$.
	\end{table}
\end{frame}




\begin{frame}[plain]
	\textbf{Note, again, that the \ding{110} signs can be \emph{anything}}.
	For us, often word counts or $tf\cdot$ idf scores ($x$) and, for supervised approaches, a topic, a sentiment, or similar ($y$). 
	
	But it could also be pixel colors or clicks on links or anything else.
	\begin{table}[]
		\resizebox{\textwidth}{!}{%
			\begin{tabular}{lllllll}
				& x1 & x2 & x3 & x4 & x5 & y \\
				case1 & \ding{110}  & \ding{110}  & \ding{110}  & \ding{110}  & \ding{110} & \ding{110} \\
				case2 & \ding{110}  & \ding{110}  & \ding{110}  & \ding{110}  & \ding{110} & \ding{110}\\
				case3 & \ding{110}  & \ding{110}  & \ding{110}  & \ding{110}  & \ding{110} & \ding{110}\\
				case4 & \ding{110}  & \ding{110}  & \ding{110}  & \ding{110}  & \ding{110} & \ding{110}\\
			\end{tabular}%
		}
	\end{table}
\end{frame}


\subsection{Zooming in on supervised ML}




\subsection{You have done it before!}
\begin{frame}{You have done it before!}
	\begin{block}{Regression}<2->
		\begin{enumerate}
			\item<3-> Based on your data, you estimate some regression equation 	$y_i = \alpha + \beta_1 x_{i1} + \cdots + \beta_p x_{ip} + \varepsilon_i$
			\item<4-> Even if you have some \emph{new unseen data}, you can estimate your expected outcome $\hat{y}$!
			\item<5-> Example: You estimated a regression equation where $y$ is newspaper reading in days/week: $y = -.8 + .4 \times man + .08 \times age$
			\item<6-> You could now calculate $\hat{y}$ for a man of 20 years and a woman of 40 years -- \emph{even if no such person exists in your dataset}: \\
			$\hat{y}_{man20} = -.8 + .4 \times 1 + .08 \times 20 = 1.2$ \\
			$\hat{y}_{woman40} = -.8 + .4 \times 0 + .08 \times 40 = 2.4$
		\end{enumerate}
	\end{block}	
	
\end{frame}



\begin{frame}{}
	\huge{This is\\ Supervised Machine Learning!}
\end{frame}

\begin{frame}{\ldots but\ldots}
	\begin{itemize}
		\item<1-> We will only use \emph{half} {\tiny{(or another fraction)}} of our data to estimate the model, so that we can use the other half to check if our predictions match the manual coding (``labeled data'',``annotated data'' in SML-lingo)
		\begin{itemize}
			\item<2->e.g., 2000 labeled cases, 1000 for training, 1000 for testing --- if successful, run on 100,000 unlabeled cases
		\end{itemize}
		\item<3-> We use many more independent variables (``features'')
		\item<4-> Typically, IVs are word frequencies (often weighted, e.g. tf$\times$idf) ($\Rightarrow$BOW-representation)
	\end{itemize}
\end{frame}


\subsection{From regression to classification}

\begin{frame}[standout]
	In the machine learning world, predicting some continous value is referred to as a \textcolor{orange}{regression} task. If we want to predict a binary or categorical variable, we call it a \textcolor{orange}{classification} task.
	
	(quite confusingly, even if we use a logistic regression for the latter)
\end{frame}


\begin{frame}{Classification tasks}
	For many computational approaches, we are actually not that interested in predicting a continous value. Typical questions include:
	\begin{itemize}
		\item Is this article about topix A, B, C, D, or E?
		\item Is this review positive or negative?
		\item Does this text contain frame F?
		\item I this satire? 
		\item Is this misinformation?
		\item Given past behavior, can I predict the next click?
	\end{itemize}
\end{frame}



\begin{frame}[plain]
	\begin{columns}[]
		\column{.5\textwidth}
		
		{\tiny{http://commons.wikimedia.org/wiki/File:Precisionrecall.svg}}
		\makebox[\linewidth]{
			\includegraphics[width=\paperwidth,height=\paperheight,keepaspectratio]{../pictures/precisionrecall.png}}
		
		\column{.5\textwidth}
		\begin{block}{Some measures}
			\begin{itemize}
				\item Accuracy
				\item Recall
				\item Precision
				\item $\text{F1}=2\cdot \frac{\text{precision}\cdot \text{recall}}{\text{precision}+\text{recall}}$
				\item AUC (Area under curve) $[0,1]$, $0.5=$ random guessing
			\end{itemize}
		\end{block}
		
		
	\end{columns}
	
\end{frame}





\begin{frame}{Different classification algorithms}
	
	\begin{itemize}[<+->]
		\item It is an empirical question which one works best
		\item We typically try several ones and select the best
		\item (remember: we have a test dataset that we did \emph{not} use to train the model, so that we can assess how well it predicts the test labels based on the test features)
		\item To avoid $p$-hacking-like scenario's (which we call ``overfitting''), there are techniques available (cross-validation, later in this course)
	\end{itemize}
	(to make it easier, imagine a binary classfication ("positive"/"negative"), but it doesn't really matter whether there are two or more labels)
\end{frame}






\begin{frame}{Naïve Bayes}
	\begin{block}{Bayes' theorem}
		$$ P(A \mid B) = \frac{P(B \mid A) \times P(A)}{P(B)} $$
	\end{block}
	\pause
	\textcolor{red}{A = Text is about sports\\
		B = Text contains `very', `close', `game'}.
	\pause
	Furthermore, we simply multiply the propabilities for the features:
	\textcolor{red}{$$P(B) = P(very\, close\, game) = P(very) \times P(close) \times P(game)$$}
	We can fill in all values by counting how many articles are about sports, and how often the words occur in these texts.
	\vspace{0.3cm}
	\footnotesize{
		(Fully elaborated example on \url{https://monkeylearn.com/blog/practical-explanation-naive-bayes-classifier/})}
\end{frame}

\begin{frame}{Naïve Bayes}
	\begin{itemize}[<+->]
		\item It's ``naïve'' because the features are treated as completely independent ($\neq$ ``controlling'' in regression analysis)
		\item It's fast and easy
		\item It's a good \emph{baseline} for binary classification problems
	\end{itemize}
\end{frame}




\begin{frame}{Na\"ive Bayes}
	$$ P(\rm{label} \mid \rm{features}) =$$
	$$ \frac{P(x_1 \mid label) \cdot P(x_2 \mid \rm{label})\ \cdot P(x_3 \mid label) \cdot P(label)}{P(x_1) \cdot P(x_2) \cdot P(x_3)}$$.
	
	
	\begin{itemize}
		\item Formulas always look intimidating, but we only need to fill in how many documents containing feature $x_n$ have the label, how often the label occurs, and how often each feature occurs
		\item Also for computers, this is \emph{really easy and fast}
		\item Weird assumption: features are independent
		\item Often used as a baseline
	\end{itemize}
\end{frame}




\begin{frame}{Logistic Regression}
	\begin{block}{Probability of a binary outcome in a regression model}
		$$p = \frac{1}{1 + e^{-(\beta_0 + \beta_1 x_1 + \beta_2 x_2 + \ldots + \beta_n x_n)}}$$
	\end{block}
	Just like in OLS regression, we have an intercept and regression coefficients. 
	We use a threshold (default: 0.5) and above, we assign the positive label (`good movie'), below, the negative label (`bad movie').
\end{frame}
\begin{frame}{Logistic Regression}
	\begin{itemize}[<+->]
		\item The features are \emph{not} independent.
		\item Computationally more expensive than Naïve Bayes
		\item We can get probabilities instead of just a label
		\item That allows us to say how sure we are for a specific case
		\item \ldots or to change the threshold to change our precision/recall-tradeoff
	\end{itemize}
\end{frame}

\begin{frame}{Support Vector Machines}
	\begin{columns}
		\column{.5\linewidth}
		\begin{itemize}
			\item	Idea: Find a hyperplane that best seperates your cases
			\item Can be linear, but does not have to be (depends on the so-called kernel you choose)
			\item Very popular 
		\end{itemize}
		\column{.5\linewidth}	
		\includegraphics[width=.8\linewidth,height=.5\paperheight,keepaspectratio]{../pictures/svm}
		\tiny{\url{https://upload.wikimedia.org/wikipedia/commons/b/b5/Svm\_separating\_hyperplanes\_\%28SVG\%29.svg}}
	\end{columns}
	\vfill
	\footnotesize{(Further reading: \url{https://monkeylearn.com/blog/introduction-to-support-vector-machines-svm/)}}
\end{frame}

\begin{frame}{SVM vs logistic regression}
	\begin{itemize}
		\item for \emph{linearly separable} classes not much difference
		\item with the right hyperparameters, SVM is less sensitive to outliers
		\item biggest advantage: with the \emph{kernel trick}, data can be transformed that they \emph{become} linearily separable
	\end{itemize}
\end{frame}


\begin{frame}{Decision Trees and Random Forests}
	\begin{columns}
		\column{.5\linewidth}
		\begin{itemize}[<+->]
			\item Model problem as a series of decisions (e.g., if cloudy then \ldots if temperature > 30 degrees then \ldots)
			\item Order and cutoff-points are determined by an algorithm
			\item Big advantage: Model non-linear relationships
			\item And: They are easy to interpret (!) (``white box'')
		\end{itemize}
		\column{.5\linewidth}	
		\includegraphics[width=.8\linewidth,height=.5\paperheight,keepaspectratio]{../pictures/decisiontree}
		\tiny{\url{https://upload.wikimedia.org/wikipedia/en/4/4f/GEP\_decision\_tree\_with\_numeric\_and\_nominal\_attributes.png}}
	\end{columns}
\end{frame}
\begin{frame}{Decision Trees and Random Forests}
	\begin{block}{Disadvantages of decision trees}
		\begin{itemize}
			\item comparatively inaccurate
			\item once you are in the wrong branch, you cannot go `back up'
			\item prone to overfitting (e.g., outlier in training data may lead to completely different outcome)
		\end{itemize}
	\end{block}
	\pause
	Therfore, nowadays people use \emph{random forests}: Random forests \emph{combine} the predictions of \emph{multiple} trees
	$\Rightarrow$ might be a good choice for your non-linear classification problem
\end{frame}











\subsection{Our first machine learning model in scikit-learn}

\begin{frame}[plain]
\instruction{go to notebook, show scikit-learn}

\instruction{in case people are interested, show gensim for LDA (even though a bit off-topic)}

\end{frame}






\begin{frame}[plain]
	\printbibliography
\end{frame}


\end{document}
